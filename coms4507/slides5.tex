% !TeX spellcheck = en_US
% !TeX root = notes.tex
\section{HashGraph}
\begin{itemize}
	\item HashGraph boasts itself as an alternative way to achieve a distributed ledger
	\item HashGraph becomes a `graph' by not excluding forks (as a Blockchain would)
	\item Using the Gossip protocol, the HashGraph `weaves' branches back into the main chain
	\item HashGraph doesn't use proof-of-work or proof-of-stake. Virtual voting is used to reach consensus
	\item Virtual Voting:
	\begin{description}
		\item[3 steps:]
			\begin{itemize}
				\item Divide rounds
				\item Decide fame
				\item Find order
			\end{itemize}
		\item[Divide rounds:]
			\begin{itemize}
				\item Two concepts, rounds and witnesses
				\item The first event for a member's node is that node's first witness
				\item The first witness is the beginning of the first round for that node
				\item All subsequent rounds are part of that first round until a new witness is discovered
				\item A witness is discovered when a node creates an event that can strongly see $\frac{2}{3}$ of the witnesses in the current round
			\end{itemize}
		\item[Decide fame:]
			\begin{itemize}
				\item A witness must be either a famous witness, or not
				\item If many of the witnesses in the next round can see a witness, it has a high chance of being famous
				\item Votes are cast on the fame of witnesses ($\frac{2}{3}$ of votes need to be cast to determine a witness' fame or infamy)
			\end{itemize}
		\item[Find order:]
			\begin{itemize}
				\item Events are sorted into either before the famous witness was decided, or after the famous witness was decided
			\end{itemize}
	\end{description}
	\item Absolute unique order to all transactions
\end{itemize}