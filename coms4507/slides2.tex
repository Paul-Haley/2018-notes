% !TeX spellcheck = en_US
% !TeX root = notes.tex
\section{Proof of Stake}
\begin{itemize}
	\item Proof-of-stake improves upon proof-of-work by providing greater security, reduced risk of centralization and drastic improvements to energy efficiency
	\item In PoS, the creator of the next block is decided randomly, and requires candidates to raise a `stake' of the cryptocurrency in order to be considered
	\item This stake is known as a bond, and it is used as a collateral to vouch for a block
	\item In PoW you know a chain with the highest collateral (higher value of bonds)
	\item An idea present in this implementation is that people who want to create the next block are invested into the platform and the cryptocurrency, as they are providing a stake of their commitment. In proof-of-work, a miner only has to solve the puzzle
	\item The `Nothing at Stake' problem is an issue initially faced in Peercoin, where there was only rewards given for creating blocks, and no punishments. This creates the situation where a validator, in the case of a forked blockchain, is incentivized to build upon both of the forks, rather than one or the other (which would be the case in proof-of-work, because the miner has to commit to solving the puzzle for a particular fork, a miner does not have the luxury of choosing)
	\item A solution to the `Nothing at Stake' problem is to punish a validator for creating a block on two separate chains, by deducting the value in their stake, essentially fining them for bad behavior
	\item The `Long-Range Attack' vulnerability is another issue with Proof of Stake
	\item The long range attack can be done if there is an attacker with 1\% of all coins at or shortly after the genesis block. that attacker then starts their own chain, and starts mining it. Although the attacker will find themselves selected for producing a block only 1\% of the time, they can easily produce 100 times as many blocks, and simply create a longer blockchain in that way
\end{itemize}