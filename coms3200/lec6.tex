% !TeX spellcheck = en_US
% !TeX root = notes.tex
\subsection{TCP Segment Structure}\label{sec:tcp}
\begin{leftbar}
	TCP contains a handshake to make sure both ends are willing to open a connection
\end{leftbar}
\begin{table}[H]
	\centering
	\caption{TCP Segment Structure}
	\begin{tabular}{cc}
		\toprule
		\multicolumn{2}{c}{32 bits}\\
		\midrule
		source port \# & dest port \#\\
		\multicolumn{2}{c}{sequence number}\\
		\multicolumn{2}{c}{acknowledgment number}\\
		(head len, not used, UAPRSF) & receive window\\
		checksum & urg data pointer\\
		\multicolumn{2}{c}{options (variable length)}\\
		\multicolumn{2}{c}{application data (variable length)}\\
		\bottomrule
	\end{tabular}
\end{table}
\begin{leftbar}
	\textbf{sequence number, acknowledgment number:} counting by bytes of data (not segments)\\
	\textbf{U:} urgent data (generally not used)\\
	\textbf{A:} ACK \# valid\\
	\textbf{P:} push data now (generally not used)\\
	\textbf{RSF:} RST, SYN, FIN; connection established (setup, teardown commands)\\
	\textbf{checksum:} Internet checksum (as in UDP)\\
	\textbf{receive window:} \# bytes receiver willing to accept
\end{leftbar}
\begin{description}
	\item[Sequence Numbers:] byte stream ``number'' of first byte in segment's data
	\item[Acknowledgements:] sequence \# of next byte expected from other side, cumulative ACK
\end{description}
\subsubsection{TCP Round Trip Time, Timeout}
$$ E = (1-\alpha)\times E+\alpha\times\textit{SampleRTT} $$
Where $E$ is EstimatedRTT. Influence of past sample decreases exponentially fast. Typical value: $\alpha = 0.125$
$$ \textit{TimeoutInterval} = E + 4\times\textit{DevRTT} $$
Where \textit{DevRTT} is the safety margin ($ \textit{DevRTT} = (1-\beta)\times\textit{DevRTT}+\beta\times\mid\textit{SampleRTT}-E\mid $(typically, $\beta=0.25$))
\subsubsection{TCP Flow Control}
\begin{itemize}
	\item receiver ``advertises'' free buffer space by including \texttt{rwnd} value in TCP header of receiver-to-sender segments
	\begin{itemize}
		\item \texttt{RcvBuffer} size set via socket options (typical default is 4096 bytes)
		\item many operating systems autoadjust \texttt{RcvBuffer}
	\end{itemize}
	\item sender limits amount of unacked (``in-flight'') data to receiver's \texttt{rwnd} value
	\item guarantees receive buffer will not overflow
\end{itemize}
\subsubsection{Closing}
\begin{itemize}
	\item client, server each close their side of connection (send TCP segment with FIN bit 1)
	\item respond to received FIN with ACK (on receiving FIN, ACK can be combined with own FIN)
	\item simultaneous FIN exchanges can be handled
\end{itemize}

\subsection{TCP Congestion Control}
\textbf{Approach:} sender increases transmission rate (window size), probing for usable bandwidth, until loss occurs
\begin{description}
	\item[Additive Increase:] increase \texttt{cwnd} by 1 MSS every RTT until loss detected
	\item[Multiplicative Decrease:] cut \texttt{cwnd} in half after loss
\end{description}

\subsection{Fairness}
TCP is fair because:
\begin{itemize}
	\item additive increase gives slope of 1, as throughput increases
	\item multiplicative decrease decreases throughput proportionally
\end{itemize}
UDP is not fair:
\begin{itemize}
	\item do not want rate throttled by congestion control
	\item send audio/video at constant rate, tolerate packet loss
\end{itemize}

\subsection{Explicit Congestion Notification}
Network-assisted Congestion Control:
\begin{itemize}
	\item two bits in IP header (ToS field) marked by \textbf{network router} to indicated congestion
	\item congestion indication carried to receiving host
	\item receiver (seeing congestion indication in IP datagram) sets ECE bit on receiver-to-sender ACK segment to notify sender of congestion
\end{itemize}