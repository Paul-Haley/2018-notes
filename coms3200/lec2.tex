% !TeX spellcheck = en_US
% !TeX root = notes.tex
\subsection{Application Architectures}
\subsubsection{Client-Server}
\begin{description}
	\item[Server:] Always-on host, Permanent IP address
	\item[Clients:] Do not communicate directly with each other, May have dynamic IP addresses
\end{description}

\subsubsection{Peer-to-Peer (P2P)}
\begin{itemize}
	\item No always-on server
	\item Peers request service from other peers, provide service in return to other peers
	\item \textbf{Self Scalability} -- new peers bring new service capacity, as well as new service demands
	\item Pers are intermittently connected and change IP addresses
\end{itemize}

\begin{note}{App-layer protocol defines}
	\begin{itemize}
		\item \textbf{type of messages exchanged} -- e.g. request, response
		\item \textbf{message syntax} -- what fields in messages and how fields are delineated
		\item \textbf{message semantics} -- meaning of information in fields
		\item \textbf{rules} for when and how processes send and respond to messages
		\item \textbf{open protocols} -- defined in RFCs, allows for interoperability (e.g. HTTP, SMTP)
		\item \textbf{proprietary protocols} -- e.g. Skype
	\end{itemize}
\end{note}

\subsection{Transport Service is needed}
\begin{description}
	\item[Data Integrity:] Some programs need 100\% reliable data transfer (e.g. file transfer, web transactions), others can tolerate loss (e.g. audio)
	\item[Timing:] Some programs require low delay to be ``effective'' (e.g. online games)
	\item[Throughput:] Some programs require minimum amount of throughput to be ``effective'' (e.g. multimedia), some use whatever they have available (``elastic apps'')
	\item[Security:] Encryption, Data Integrity
\end{description}

\subsection{Transport Protocol Services}
\subsubsection{TCP}
\begin{description}
	\item[Reliable Transport] between sending and receiving process
	\item[Flow Control:] sender won't overwhelm receiver
	\item[Congestion Control:] throttle sender when network overloaded
	\item[Connection-Oriented:] setup required between client and server processes
	\item[Does Not Provide:] timing, minimum throughput guarantee, security
\end{description}

\subsubsection{UDP}
\begin{description}
	\item[Unreliable Data Transfer] between sending and receiving process
	\item[Does Not Provide:] reliability, flow control, congestion control, timing, throughput guarantee, security, or connection setup
\end{description}

\subsubsection{Securing TCP}
\textbf{TCP and UDP}
\begin{itemize}
	\item no encryption
	\item cleartext passwords sent into socket traverse Internet in cleartext
\end{itemize}
\textbf{SSL}
\begin{itemize}
	\item provides encrypted TCP connection
	\item data integrity
	\item end-point authentication
\end{itemize}
\textbf{SSL is at app layer}
\begin{itemize}
	\item app use SSL libraries, that ``talk'' to TCP
\end{itemize}
\textbf{SSL socket API}
\begin{itemize}
	\item cleartext passwords sent into socket traverse Internet encrypted
\end{itemize}

\subsection{HTTP: Hypertext Transfer Protocol}
\begin{itemize}
	\item Web's application layer protocol
	\item client/server model. Client request website and server serves HTTP object in response
	\item Uses TCP
	\item HTTP is stateless. Server maintains no information about past client requests
	\item \textbf{non-persistent HTTP:} one object sent over one TCP connection, downloading multiple object required multiple connections
	\item \textbf{persistent HTTP:} multiple object sent over single TCP connection
\end{itemize}
Non-persistent HTTP issues:
\begin{itemize}
	\item requires 2 RTTs per object
	\item OS overhead for each TCP connection
	\item browsers often open parallel TCP connections to fetch referenced objects
\end{itemize}
Persistent HTTP:
\begin{itemize}
	\item server leaves connection open after sending response
	\item subsequent HTTP messages between same client/server sent over open connection
	\item client sends requests as soon as it encounters a referenced object
	\item as little as one RTT for all the referenced objects
\end{itemize}
\subsubsection{Method Types}
\textbf{HTTP/1.0:} GET, POST, HEAD (asks server to leave requested object out of response)\\
\textbf{HTTP/1.1:} GET, POST, HEAD, PUT (uploads file in entity body to path specified in URL field), DELETE (deletes file specified in the URL field)
\subsubsection{Response Codes}
\begin{description}
	\item[200 OK:] request succeeded, requested object later in this msg
	\item[301 Moved Permanently:] requested object moved, new location specified later in this msg
	\item[400 Bad Request:] request msg not understood by server
	\item[404 Not Found:] requested document not found on this server
	\item[505 HTTP Version Not Supported]
\end{description}

\subsection{Cookies}
Uses: authorization, shopping carts, recommendations, user session state (Web, email)

\subsection{Web Caches (proxy server)}
\begin{leftbar}
	\textbf{Goal:} satify client request without involving origin server
\end{leftbar}
\begin{itemize}
	\item Browsers requests object from cache, if in cache the object is sent back otherwise cache requests object from origin
	\item Cache acts as both client and server
	\item Reduce response time for client request
	\item Reduce traffic
\end{itemize}

\subsubsection{Conditional GET}
\begin{leftbar}
	\textbf{Goal:} don't send object if cache has up-to-date cached version (lower link usage)
\end{leftbar}
\begin{itemize}
	\item \textbf{Cache:} specify date of cached copy in HTTP request \texttt{If-modified-since: <date>}
	\item \textbf{Server:} response contains no object if cached copy is up-to-date: \texttt{HTTP/1.0 Not Modified}
\end{itemize}

\subsection{Electronic Mail: SMTP}
\textit{RFC 2821}
\begin{itemize}
	\item uses TCP to reliably transfer email message from client to server, port 25
	\item direct transfer: sending server to receiving server
	\item three phases of transfer: handshaking, transfer of messages, closure
	\item command/response interaction
	\item messages must be in 7-bit ASCII
	\item uses persistent connections
	\item requires message to be in 7-bit ASCII
	\item uses \texttt{CRLF.CRLF} to determine end of message
\end{itemize}
Difference to HTTP being, HTTP is server sending data, SMTP is client connection sending data\\
\begin{description}
	\item[SMTP:] protocol for exchanging email messages
	\item[RFC 822:] standard for text message format (To, From, Subject, Body)
\end{description}
\subsubsection{Mail Access Protocols}
\begin{leftbar}
	\textbf{SMTP:} delivery/storage to receiver's server
\end{leftbar}
\begin{description}
	\item[POP:] Post Office Protocol \textit{(RFC 1939)}: authorization, download
	\begin{itemize}
		\item POP3 is stateless across sessions
		\item Two main modes; download and delete, download and keep (allows multiple clients to read the same email)
	\end{itemize}
	\item[IMAP:] Internet Mail Access Protocol \textit{(RFC 1730)}: more features, including manipulation of stored message on server
	\begin{itemize}
		\item All messages stored on server
		\item Supports folders
		\item Keeps user state across sessions: names of folders and mappings between message IDs and folder name
	\end{itemize}
	\item[HTTP:] gmail, Hotmail, Yahoo, etc
\end{description}

\subsection{DNS: Domain Name System}
\begin{itemize}
	\item Lookup between names (e.g. google.com) and IP addresses
	\item \textbf{Distributed Database} implemented in hierarchy of many \textbf{name servers}
	\item \textbf{Application-layer protocol:} hosts, name servers communicate to \textbf{resolve} names (address/name translation)
\end{itemize}
Why not centralize DNS? Single point of failure, traffic volume, doesn't scale
\subsubsection{DNS Services}
\begin{itemize}
	\item hostname to IP address translation
	\item host aliasing (canonical, alias names)
	\item mail server aliasing
	\item load distribution (many IP addresses correspond to one name)
\end{itemize}
\subsubsection{TLD, authoritative servers}
\textbf{top-level domain (TLD) servers:}
\begin{itemize}
	\item responsible for com, org, net, edu, aero, jos, io
	\item and top-level country domains au, uk, ca
	\item Network Solutions maintains servers for .com TLD
	\item Educause for .edu TLD
\end{itemize}
\textbf{Authoritative DNS servers:}
\begin{itemize}
	\item organization's own DNS server(s), providing authorative hostname to IP mappings for organization's named hosts
	\item can be maintained by organization or service provider
\end{itemize}
\subsubsection{Local DNS name server}
\begin{itemize}
	\item does not strictly belong to hierarchy
	\item each ISP (residential ISP, company, university) has one (also called ``default name server'')
	\item when host makes DNS query, query is sent to its local DNS server
	\begin{itemize}
		\item has local cache of recent name-to-address translation pairs (but may be out of date!)
		\item acts as proxy, forwards query into hierarchy
	\end{itemize}
\end{itemize}
\subsubsection{DNS Name Resolution}
\begin{description}
	\item[Iterated query:] contacted server replies with name of server to contact. So root dns sends the ip of the next dns server to contact
	\item[Recursive query:] puts burden of name resolution on contacted name server. So root dns server contacts the next levels down which contacts next level down.
\end{description}
\subsubsection{Caching}
Once (any) name server learns mapping, it \textbf{caches} mapping. Cache entries timeout (disappear) after some time (TTL). If name host changes IP address, the name servers might not update until TTLs expire.
\begin{leftbar}
	update/notify mechanisms proposed IETF standard RFC 2136
\end{leftbar}
\subsubsection{DNS Records}
\begin{note}{RR Format}
	\texttt{(name, value, type, ttl)}
\end{note}
\begin{description}
	\item[type=A] \texttt{name} is hostname, \texttt{value} is IP address
	\item[type=NS] \texttt{name} is domain (e.g. google.com), \texttt{value} is hostname of authoritative name server for this domain
	\item[type=CNAME] \texttt{name} is alias name for some ``canonical'' (the real) name (\texttt{www.ibm.com} is really \texttt{servereast.backup2.ibm.com}), \texttt{value} is canonical name
	\item[type=MX] \texttt{value} is name of mailserver associated with \texttt{name}
\end{description}
\subsubsection{Protocol}
Query and reply messages both follow same format
\begin{table}[H]
\centering
\caption{Protocol Layout}
\begin{tabular}{cc}
	\toprule
	2 bytes & 2 bytes \\
	\midrule
	identification & flags\\
	\# questions & \# answer RRs\\
	\# authority RRs & \# additional RRs\\
	\multicolumn{2}{c}{questions (variable \# of questions)}\\
	\multicolumn{2}{c}{answers (variable \# of RRs)}\\
	\multicolumn{2}{c}{authority (variable \# of RRs)}\\
	\multicolumn{2}{c}{additional info (variable \# of RRs)}\\
	\bottomrule
\end{tabular}
\end{table}
\subsubsection{Attacking DNS}
\textbf{DDoS attacks}
\begin{itemize}
	\item bombard root servers with traffic. Not successful to date, traffic filtering, local DNS servers cache protecting root DNS
	\item bombard TLD server. Potentially more dangerous
\end{itemize}
\textbf{Redirect Attacks}
\begin{itemize}
	\item man-in-middle (Intercept queries)
	\item DNS Poisoning (Send bogus relies to DNS server, which caches)
\end{itemize}
\textbf{Exploit DNS for DDoS}
\begin{itemize}
	\item send queries with spoofed source address: target IP
	\item requires amplification
\end{itemize}